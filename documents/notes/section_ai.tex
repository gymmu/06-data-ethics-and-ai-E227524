\section{Künstliche Intelligenz}
\label{sec:ai}

Einleitung
Was versteht man unter künstlicher Intelligenz?
KI, kurz für Künstliche Intelligenz, beschreibt Technologien, die intelligentes Verhalten imitieren, zu dem bisher nur Menschen fähig waren.
Beispiele für KI-Fähigkeiten:
Lernen und Problemlösen: KI-Systeme können aus Daten lernen, Muster erkennen und Probleme eigenständig lösen.
Entscheidungen treffen: KI kann analysieren und Informationen verarbeiten, um fundierte Entscheidungen zu treffen.
Sprechen und Verstehen: KI kann Sprache verstehen und selbst sprechen, sogar in verschiedenen Sprachen.
Kreative Aufgaben: KI kann kreative Aufgaben wie Text schreiben, Musik komponieren oder Bilder erstellen.
KI in unserem Alltag:
KI ist bereits heute in vielen Bereichen unseres Alltags präsent, oft ohne dass wir es bemerken.
Empfehlungssysteme: KI hilft uns, Produkte und Inhalte zu finden, die uns interessieren könnten, z.B. in Online-Shops oder Streaming-Diensten.
Virtuelle Assistenten: KI-basierte Sprachassistenten wie Siri oder Alexa können uns bei alltäglichen Aufgaben unterstützen.
Selbstfahrende Autos: KI ermöglicht die Entwicklung von selbstfahrenden Autos, die den Straßenverkehr sicherer und effizienter machen könnten.
Medizinische Diagnosen: KI kann Ärzten bei der Diagnose von Krankheiten unterstützen, indem sie Muster in medizinischen Bildern und Daten erkennt.
Wichtig: KI befindet sich noch in der Entwicklung und es gibt viele Herausforderungen, die gemeistert werden müssen.
Es ist wichtig, dass KI verantwortungsvoll und ethisch eingesetzt wird, um sicherzustellen, dass sie dem Menschen dient und nicht schadet.
Weitere Informationen:
https://de.wikipedia.org/wiki/K%C3%BCnstliche_Intelligenz
https://digitalzentrum-augsburg.de/
https://weissenberg-group.de/thema/kuenstliche-intelligenz/


Wie funktioniert künstliche Intelligenz?
Wie funktioniert KI
KI, oder Künstliche Intelligenz, ist ein weites Feld mit verschiedenen Ansätzen. Um Ihnen die Funktionsweise zu erklären, muss ich etwas ausholen.
Grundprinzip:
KI-Systeme versuchen, menschliche Intelligenz in Maschinen nachzubilden. Sie sollen Aufgaben lernen und ausführen können, die bisher nur Menschen vorbehalten waren. Dazu nutzen sie verschiedene Methoden, wie zum Beispiel:
Maschinelles Lernen: Hierbei werden Algorithmen mit Daten trainiert, um Muster zu erkennen und Vorhersagen zu treffen. Ein Beispiel ist die Bildersuche, bei der ein Algorithmus anhand von Trainingsbildern neue Bilder erkennen kann.
Natürliche Sprachverarbeitung: Damit können KI-Systeme menschliche Sprache verstehen und erzeugen. Dies ermöglicht Anwendungen wie Chatbots oder Spracherkennung.
Künstliche neuronale Netze: Diese Systeme sind vom menschlichen Gehirn inspiriert und können komplexe Zusammenhänge lernen. Sie werden unter anderem in der Bilderkennung und Spracherkennung eingesetzt.
]
Arten von KI:
Man unterscheidet zwei Hauptkategorien von KI:
Schwache KI: Diese Systeme sind auf die Lösung spezifischer Probleme spezialisiert. Sie können zum Beispiel Schach spielen oder Gesichter erkennen.
Starke KI: Diese hypothetische KI würde die menschliche Intelligenz in allen Bereichen übertreffen. Sie ist derzeit noch Zukunftsvision.
Anwendung:
KI wird bereits heute in vielen Bereichen eingesetzt, zum Beispiel:
Medizin: KI kann bei der Diagnose von Krankheiten, der Entwicklung neuer Medikamente und der Durchführung von Operationen helfen.
Verkehr: KI kann autonome Fahrzeuge entwickeln, den Verkehr optimieren und Unfälle reduzieren.
Wirtschaft: KI kann Unternehmen bei der Automatisierung von Prozessen, der Kundenanalyse und der Betrugsprävention helfen.


Kritik:
Der Einsatz von KI birgt auch Risiken, wie zum Beispiel:
Jobverluste: Durch die Automatisierung von Prozessen könnten Arbeitsplätze verloren gehen.
Diskriminierung: KI-Systeme können Vorurteile ihrer Entwickler widerspiegeln und diskriminierende Entscheidungen treffen.
Machtmissbrauch: KI könnte in den falschen Händen zu Machtmissbrauch und Überwachung führen.
Fazit:
KI ist eine mächtige Technologie mit dem Potenzial, unser Leben in vielerlei Hinsicht zu verbessern. Es ist jedoch wichtig, die Risiken zu kennen und die Entwicklung und den Einsatz von KI verantwortungsvoll zu gestalten.
Weitere Informationen:
https://de.wikipedia.org/wiki/K%C3%BCnstliche_Intelligenz
https://www.scopevisio.com/
https://www.ibm.com/de-de/artificial-intelligence

Deep Learning, auch als tiefes Lernen bezeichnet, ist ein Teilbereich des maschinellen Lernens, der neuronale Netze verwendet, um komplexe Aufgaben zu bewältigen. Inspiriert vom Aufbau und der Funktionsweise des menschlichen Gehirns, nutzt Deep Learning künstliche neuronale Netze, die Algorithmen sind, die lose an den Netzwerken von Neuronen im Gehirn angelehnt sind.
Die Funktionsweise von Deep Learning:
1.Datenvorbereitung: Daten werden in die erste Schicht des neuronalen Netzes eingespeist. Dies können Bilder, Texte, Zahlen oder andere Arten von Informationen sein.
2.Verarbeitung Schicht für Schicht: Jede Schicht im Netzwerk wendet mathematische Funktionen auf die Daten an, transformiert sie und extrahiert immer komplexere Merkmale. Die "Tiefe" im Deep Learning bezieht sich auf die Verwendung mehrerer Schichten, die es dem Netzwerk ermöglichen, komplexe Beziehungen in den Daten zu lernen.
3.Lernen und Optimierung: Während des Trainings wird dem Netzwerk beschriftete Daten (Daten mit bekannten Ausgaben) präsentiert. Das Netzwerk vergleicht seine Vorhersagen mit den richtigen Beschriftungen und passt seine internen Parameter (Gewichte und Biases) an, um den Fehler zu minimieren. Dieser Prozess wird iterativ wiederholt, so dass das Netzwerk lernen und seine Leistung verbessern kann.
4.Vorhersagen treffen: Sobald ein Deep-Learning-Modell trainiert ist, kann es verwendet werden, um Vorhersagen für neue, unsichtbare Daten zu treffen. Dazu verarbeitet es die Daten durch die Schichten und generiert eine Ausgabe basierend auf dem, was es gelernt hat.
Vorteile von Deep Learning:
Hohe Genauigkeit: Deep-Learning-Modelle können bei vielen Aufgaben, einschließlich Bilderkennung, Verarbeitung natürlicher Sprache und Spracherkennung, eine Leistung auf dem neuesten Stand der Technik erzielen.
Fähigkeit, aus großen Datensätzen zu lernen: Deep-Learning-Algorithmen eignen sich besonders gut für die Verarbeitung großer Datenmengen, was in der heutigen datengetriebenen Welt ein erheblicher Vorteil sein kann.
Automatische Merkmalsextraktion: Deep-Learning-Modelle können automatisch Merkmale aus den Daten lernen, wodurch die manuelle Merkmalstechnik entfällt, die ein zeitaufwändiger und mühsamer Prozess sein kann.


Anwendungen von Deep Learning:
Computer Vision: Deep Learning wird in Anwendungen wie Gesichtserkennung, Objekterkennung und selbstfahrenden Autos eingesetzt.
Verarbeitung natürlicher Sprache: Deep Learning ermöglicht Anwendungen wie maschinelle Übersetzung, Chatbots und Sentimentanalyse.
Spracherkennung: Deep Learning wird in Sprachassistenten wie Siri und Alexa sowie in automatischen Transkriptionsdiensten eingesetzt.
Empfehlungssysteme: Deep Learning wird verwendet, um personalisierte Empfehlungen für Produkte, Filme, Musik und mehr zu erstellen.
Medikamentenentwicklung: Deep Learning wird eingesetzt, um die Medikamentenentwicklung zu beschleunigen, indem große Datensätze von Molekülen und ihren Eigenschaften analysiert werden.
Herausforderungen von Deep Learning:
Rechenaufwand: Das Training von Deep-Learning-Modellen kann erhebliche Rechenressourcen erfordern, z. B. leistungsstarke GPUs.
Datenanforderungen: Deep-Learning-Modelle benötigen in der Regel große Datenmengen, um effektiv trainiert werden zu können. Dies kann für Aufgaben eine Herausforderung sein, bei denen die Daten begrenzt sind.
Erklärbarkeit: Deep-Learning-Modelle können komplex und schwer zu verstehen sein, was es schwierig macht, zu erklären, wie sie zu ihren Vorhersagen kommen.
Insgesamt ist Deep Learning ein sich schnell entwickelndes Gebiet mit dem Potenzial, viele Aspekte unseres Lebens zu revolutionieren. Mit der Zunahme der Rechenleistung und der leichteren Verfügbarkeit von Daten können wir in Zukunft noch innovativere Anwendungen von Deep Learning erwarten.
Weitere Informationen:
https://en.wikipedia.org/wiki/Deep_learning
https://www.youtube.com/c/deeplearningai
[https://sebastianraschka.com/blog/2021/dl-course.html]
[https://datasolut.com/category/deep-learning-grundlagen/]





Geschichtlicher Hintergrund KI:
Die Geschichte der Künstlichen Intelligenz: Ein kurzer Überblick
Die Geschichte der Künstlichen Intelligenz (KI) ist eng mit der Entwicklung der Computertechnologie verbunden. Bereits in den 1940er Jahren begannen Forscher, sich mit der Möglichkeit zu beschäftigen, Maschinen zu bauen, die "denken" können.
1950: Alan Turing veröffentlichte sein einflussreiches Werk "Computing Machinery and Intelligence", in dem er den Turing-Test als Maßstab für die Intelligenz von Maschinen vorschlug.
1956: Auf der Dartmouth-Konferenz wurde der Begriff "Künstliche Intelligenz" (KI) erstmals verwendet.
1960er Jahre: Die ersten KI-Programme wurden entwickelt, die einfache Aufgaben wie das Lösen von Rätseln und das Spielen von Spielen ausführen konnten.
1970er Jahre: Die Entwicklung von KI erlebte einen Rückgang, da sich die Forschung auf andere Bereiche der Informatik konzentrierte.
1980er Jahre: Mit der Entwicklung neuer Computertechnologien und Algorithmen erlebte die KI-Forschung eine Renaissance.
1990er Jahre: Das Aufkommen des Internets führte zu neuen Anwendungen für KI, wie z. B. Suchmaschinen und Spracherkennung.
2000er Jahre: Die Entwicklung von "Deep Learning" führte zu einem neuen Durchbruch in der KI-Forschung. Deep-Learning-Algorithmen können aus grossen Datenmengen lernen und komplexe Aufgaben wie Bild- und Spracherkennung ausführen.
2010er Jahre: KI-Technologien wurden in vielen Bereichen des täglichen Lebens eingesetzt, z. B. in Smartphones, Autos und Haushaltsgeräten.
2020er Jahre: Die KI-Forschung entwickelt sich weiterhin rasant und es werden ständig neue Anwendungen für KI entwickelt.
Meilensteine in der Geschichte der KI:
1956: Entwicklung des General Problem Solvers (GPS) durch Allen Newell und Herbert Simon, eines der ersten Programme, das Probleme in natürlicher Sprache lösen konnte.
1965: Entwicklung des ELIZA-Programms durch Joseph Weizenbaum, eines der ersten Chatbots.
1970: Entwicklung des SHRDLU-Programms durch Terry Winograd, eines der ersten Programme, das natürliche Sprache verstehen und generieren konnte.
1997: Der IBM-Computer Deep Blue besiegt den Schachweltmeister Garry Kasparov.
2011: Der IBM-Computer Watson gewinnt die Quizshow Jeopardy!
2016: Das AlphaGo-Programm von DeepMind besiegt den Go-Weltmeister Lee Sedol.
2018: Das GPT-3-Sprachmodell von OpenAI wird veröffentlicht, das menschenähnliche Texte generieren kann.
Heute wird KI in einer Vielzahl von Bereichen eingesetzt, darunter:
Gesundheitswesen: KI wird zur Diagnose von Krankheiten, zur Entwicklung neuer Medikamente und zur Personalisierung der Behandlung eingesetzt.
Finanzen: KI wird zur Betrugserkennung, zur Risikoberechnung und zur Entwicklung neuer Finanzprodukte eingesetzt.
Verkehrswesen: KI wird zur Entwicklung von selbstfahrenden Autos, zur Optimierung des Verkehrsflusses und zur Verbesserung der Sicherheit im Strassenverkehr eingesetzt.
Produktion: KI wird zur Optimierung von Produktionsprozessen, zur Qualitätskontrolle und zur Entwicklung neuer Produkte eingesetzt.
Kundenservice: KI wird zur Beantwortung von Kundenanfragen, zur Lösung von Problemen und zur Verbesserung der Kundenzufriedenheit eingesetzt.
Die Zukunft der KI:
Die KI-Forschung entwickelt sich weiterhin rasant und es ist schwierig vorherzusagen, wie sich die KI in den kommenden Jahren entwickeln wird. Es ist jedoch wahrscheinlich, dass KI in Zukunft eine noch grössere Rolle in unserem Leben spielen wird. Es ist wichtig, dass wir uns mit den ethischen und gesellschaftlichen Herausforderungen der KI auseinandersetzen, um sicherzustellen, dass sie zum Wohle aller eingesetzt wird.
Weitere Informationen:
https://de.wikipedia.org/wiki/K%C3%BCnstliche_Intelligenz
https://de.wikipedia.org/wiki/Geschichte_der_k%C3%BCnstlichen_Intelligenz
https://www.twoday.no/blogg/teknologi/en-reise-gjennom-historien-til-kunstig-intelligens
https://www.bosch-ai.com/careers/

