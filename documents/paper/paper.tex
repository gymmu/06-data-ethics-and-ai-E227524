\documentclass{report}

\usepackage[ngerman]{babel}
\usepackage[utf8]{inputenc}
\usepackage[T1]{fontenc}
\usepackage{hyperref}
\usepackage{csquotes}
\usepackage[a4paper]{geometry}

\usepackage[
    backend=biber,
    style=apa,
    sortlocale=de_DE,
    natbib=true,
    url=false,
    doi=false,
    sortcites=true,
    sorting=nyt,
    isbn=false,
    hyperref=true,
    backref=false,
    giveninits=false,
    eprint=false]{biblatex}
\addbibresource{../references/bibliography.bib}

\usepackage{graphicx}
\graphicspath{ {../images/} }

\title{Ethischer Umgang mit Daten im Zusammenhang mit der KI}
\author{Max Brügger}
\date{\today}


\newcommand{\zitieren}[1]{ - \citeauthor{#1} \citeyear{#1}}

\begin{document}

\maketitle

\tableofcontents

\section{Einleitung}

In dieser Arbeit befassen wir uns mit dem ethischen Umgang mit Daten im Zusammenhang mit der KI (künstliche Intelligenz). Zu diesem Thema  recherchieren wir im Internet mit KI-Programmen, erstellen Notizen und verfassen eine Arbeit zu einer Frage bezüglich der Ethik der KI, die wir uns stellen. Ich habe mich entschieden, das Thema KI in der Medizin zu beleuchten und die Chancen und Gefahren aufzuzeigen. 
Um die KI  besser zu verstehen, werde ich vorab kurz die KI und ihre Einsatzgebiete definieren, aufzeigen wie sie funktioniert und trainiert werden kann und einen Miniexkurs in die Geschichte der KI machen und die allgemeinen ethischen Probleme aufzeigen bevor ich im Hauptteil auf die KI in der Medizin eingehen werde.\parencite{ai-wikipedia}

\zitieren{ai-wikipedia}

\begin{figure}
    \centering
    \includegraphics[width=0.25\textwidth]{bild.png}
    \caption{Ein erstes Bild}
\end{figure}



\subsection{Was ist eine KI?}
dsfasdfasfsadf

\subsection{Wie funktioniert KI?}
lflajsdjflasdlkfjafjfas\\
asdflasdflaskdfjöasfjasfj



\subsection{Wo findet KI}

\nocite{*}
\printbibliography

\end{document}
